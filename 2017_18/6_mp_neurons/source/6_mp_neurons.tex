\documentclass[12pt]{article}
\usepackage{amsfonts, epsfig}

\usepackage{graphicx}
\usepackage{fancyhdr}
\pagestyle{fancy}
\lfoot{\texttt{github.com/conorhoughton/COMS30127}}
\lhead{Computation Neuroscience - 6\_mp\_neurons (d) - Conor}
\rhead{\thepage}
\cfoot{}
\begin{document}

\section*{McCulloch-Pitts neurons} 

The McCulloch Pitts neuron model, or Threshold Logic Unit, was
introducted in 1943 by Warren McCulloch and Walter Pitts as a
computational model of a neuronal network. Their thinking was that
neurons are joined to each other with connections of variable
strength; in the soma the inputs from other neurons are added up and
they determine the activity of the neuron in a non-linear way; they
also knew that neurons tend to ignore input up to some threshold value
before responding strongly. These properties they tried to include in
their model neurons.

Artificial neurons, of the sort used in artificial intelligance, are
described by a single dynamical variable, $x_i$ say for a neuron
labelled $i$; the value of $x_i$ is determined by the weighted input
from the other neurons:
\begin{equation}
x_i=\phi\left(\sum_j w_{ij} x_j-\theta_i\right)
\end{equation}
$\phi$ is an activation function, $\theta$ is a threshold and the
$w_{ij}$ are the connection strengths weighting the inputs from the
other neurons. The McCulloch Pitts neuron was the first example of an
artificial neuron and had a step function for $\phi$:
\begin{equation}
x_i=\left\{\begin{array}{ll}1&\sum_j w_{ij} x_j>\theta_i\\0&\mbox{otherwise}\end{array}\right.
\end{equation}
Thus, the neuron has two states, it is in the on state, $x_i=1$ if the
weighted input exceeds a threshold $\theta_i$ and an off state,
$x_i=-1$ if it doesn't; the picture you might have of how this
corresponds to the brain is that \lq{}on\rq{} corresponds to rapid
spiking and \lq{}off\rq{} to spiking at a much lower rate. The
$w_{ij}$, the connection strengths, are like the synapse strengths, a
positive $w_{ij}$ is an excitatory synapse and negative, an
inhibitory; a given neurons have have both negative and positive
out-going synapses, that is there is no restriction that says $w_{ij}$
always has the same sign for a given $j$, this is different from real
neurons where all the outgoing synapses from a given neuron are either
excitatory or inhibitory.

While it should be clear that this network has some of the properties,
very abstracted, of a neuronal network, it might not be so clear what
can be done with the neurons. When they were working, at the dawn of
the age of electronic computers, McCulloch and Pitts believed that
their neurons might form the natural unit in computer circuits, in
other words, they thought they might perform the role actually played
by logical circuits. In fact, it is still not clear if the artificial
neuron is or isn't the natural unit of computation in silicon since
they are a component in, for example, deep learning networks. In fact,
there are two major applications of McCulloch-Pitts neurons: the
perceptron and the Hopfield network; these two applications add a rule
for changing the connect strength to the original McCulloch-Pitts
neuron.

\subsubsection*{Perceptrons}

The perceptron is a machine that does supervised learning, that is, it
makes guesses, is told whether or not its guess is correct, and then
makes another guess. They were first discovered in 1957 by Frank
Rosenblatt and introduced to the world with great fanfare, it was
claimed that they would solve problems from object recognition to
consciousness: if you consider the perceptron as the forebearer of the
deep learning network, then perhaps we don't know if these claims will
be fulfilled, but we do know that the original perceptron proved quite
limited in artificial intelligence. It does, however, appear to
describe some neuronal processes, if we ignore the implementation
details.

Anyway, a percepton is made of two layers of neurons, an input layer
and an output layer of McCulloch-Pitts neurons. For simplicity lets
assume the output layer
has a single neuron. Now, if the input is given by
$\textbf{x}=(x_1,x_2,\ldots,x_n)$ the output, $y$, is
\begin{equation}
y=\phi(r)
\end{equation}
where
\begin{equation}
r=\sum_j w_j x_j-\theta
\end{equation}
and $\phi$ here is the simple Heaviside like activitation function
described above. Now for a given input, if the actual value of the
output should be $d$ the error is $d-y$. The perceptron learning rule
is to change the $w_j$ weight by an amount proportional to the error
and how much $x_j$ was \lq{}to blame\rq{} for the error:
\begin{equation}
\delta w_j=\eta (d-y) x_j
\end{equation}
and
\begin{equation}
\delta \theta =  \eta (d-y)
\end{equation}
where $\eta$ is some small learning rate and
\begin{eqnarray}
w_j&\rightarrow& w_j+\delta w_j\cr
\theta&\rightarrow& \theta+\delta\theta
\end{eqnarray}
You can see how this might work, if $x_j$ was positive and $y$ was too
big, this would make $w_j$ smaller so in future $y$ would be smaller
when it had the same input. 

In fact, the perceptron can only solve problems with a linear
classifier: if we think of the $x_i$'s as parameterizing an
$n$-dimensional space then $\sum_iw_ix_i=\theta$ is a hyperplane in
that space, so a pattern $\textbf{x}$ is classified one way or the
other according to which side of that hyperplane it lies. Thus, the
perceptron works only if there is a hyperplane dividing the data into
two, with one class of data on one side and one on the other. In fact,
if the data is linearly seperable the perceptron is guaranteed to
converge to a solution which manages this seperation. Now, there will
not be a unique hyper-plane seperating the two classes and the
perceptron won't, typically, find what you might regard as the
\lq{}best\rq{} hyperplane, where by best you might mean the line that
is, in some sense, as far from the individual data points as possible;
finding that best hyperplane is the idea behind the support vector
machine.

The perceptron learning rule can be motivated by thinking about error
minimization. Consider an error function
\begin{equation}
E=\langle(d^a-y^a)^2\rangle_a
\end{equation}
If $a$ labels lots of input, $\textbf{x}^a$, desired output $d^a$
pairs, with $y^a$ the actual output; the angle brackets are an average
over these pairs. Now consider the gradient of $E$ with $w_{i}$:
\begin{equation}
\frac{\partial E}{\partial w_{i}}=2\left\langle (d^a-y_a) \frac{\partial y^a}{\partial w_{i}}\right\rangle
\end{equation}
Now if we ignore for the moment the fact that the activiation function for the McCulloch-Pitt neuron isn't differentiable and write
\begin{equation}
y^=\phi(r)
\end{equation}
with 
\begin{equation}
r=\sum_{i}w_i x_i-\theta
\end{equation}
we have
\begin{equation}
\frac{dy^a}{dw_i}=\frac{d\phi}{dr}\frac{\partial r}{\partial w_i}
\end{equation}
we get
\begin{equation}
\frac{\partial E}{\partial w_{i}}=2\left\langle (d^a-y^a)x^a_i (\mbox{stuff involving the differential of }\phi)\right\rangle
\end{equation}




\subsection*{Hopfield network}

In contrast to a perceptron a Hopfield network is a recurrently
connected network; it is intended to perform pattern completion and
was proposed by John Hopfield in 1982, though other people had had the
idea before in different contexts. The idea behind a Hopfield network
is that you evolve the network according to the McCulloch-Pitts
relation, so, in the synchronous update version, from one iteration to
the next
\begin{equation}
\hat{x}_i=\phi\left(\sum_j w_{ij} x_j\right)
\end{equation}
and then $x_i\rightarrow \hat{x}$; that is all the nodes update using
the old values. In the most common version of a Hopfield network, the $w_{ij}$ are symmetric, that is
\begin{equation}
w_{ij}=w_{ji}
\end{equation}
The threshold values $\theta_i$ have been set to zero, this is
something you can do in a Hopfield network. In the asynchronous
scheme, the you update the nodes one-by-one, for example, after
choosing a random node.

Now, there is an \lq{}energy\rq{} associated with a Hopfield network:
\begin{equation}
E=-\frac{1}{2}\sum_{ij} w_{ij}x_ix_j
\end{equation}
and you can show than if you update a node you will reduce the energy;
roughly speaking if you update a node $x_i$ then it is more likely to
have the same sign as a connected node $x_j$ if the connection between
them is large and positive and an opposite sign if the connect is
large and negative. It can be shown that update the nodes will evolve
the system to a local minimum.

Now the question is how to create the local minima? Two versions:
\begin{equation}
w_{ij}=\frac{1}{N}\sum{\mbox{patterns}} x_ix_j
\end{equation}
or 
\begin{equation}
\delta w_{ij}=\eta (x_i-a)(x_j-a)
\end{equation}




\end{document}

